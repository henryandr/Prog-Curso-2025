\documentclass[addpoints,11pt]{exam}
\usepackage[utf8]{inputenc}
\usepackage{graphicx}
\usepackage{tcolorbox}
\usepackage{stfloats}
\usepackage{multirow}
\usepackage{array} 
\usepackage{vhistory}
\usepackage{listings}
\usepackage{verbatim}
\usepackage{multicol,caption,capt-of}
%multi-row
\usepackage{multirow}
\usepackage[left=1.5cm,right=1.5cm,top=1.5cm,bottom=1.5cm]{geometry}

\begin{document}
	
	%Encabezados y pie de página
	%----------------------------------------------------------------------------------------
	\pagestyle{headandfoot}
	\runningheadrule
	\firstpageheader{Programación}{Examen Final}{Noviembre 4, 2025}
	\runningheader{Programación}
	{Examen Final, Page \thepage\ of \numpages}
	{Noviembre 4, 2025}
	%	\firstpagefooter{Escuela de Ingenierías}{UPB}{Fac. Ing. Eléctrica y Electrónica}
	\runningfooter{Escuela de Ingenierías}{UPB}{Ingeniería Aeronáutica}
	
	\begin{center}
		\begin{tabular}{@{}m{4cm}|m{8cm}|m{5cm}|@{}}
			\cline{2-3}
			\multirow{5}{*}{\includegraphics[width=1\linewidth]{Logo-UPB-2022.png}}
			&\multicolumn{2}{c|}{\textbf{Universidad Pontificia Bolivariana - Sede  Medellín}}		\\ \cline{2-3}
			&Curso: \textbf{Programación}		&Duración: 100 minutos				\\ \cline{2-3}
			&Preparada por:\textbf{ Henry Andrade, IEo, Ph.D.}	&Prueba: Convencional				\\ \cline{2-3}
			&\multicolumn{2}{l|}{Facultad de Ingeniería Aeronáutica}					\\ \cline{2-3}
			&\multicolumn{2}{l|}{Estudiante:				}										\\ \cline{2-3} %\hline
		\end{tabular}
	\end{center}
	
	
	%----------------------------------------------------------------------------------------
	%Cuadro con instrucciones
	%----------------------------------------------------------------------------------------		
	\begin{center}
		\fbox{\fbox{\parbox{7in}{Lea cuidadosamente cada una de las preguntas antes de contestar. No se permite el uso de ningún tipo de dispositivo electrónico. Cualquier intento de copia o fraude dará inicio a un proceso disciplinario.}}}
	\end{center}
	
	%----------------------------------------------------------------------------------------
	\begin{multicols}{2}
	%----------------------------------------------------------------------------------------
	\begin{questions} 
	\boxedpoints
	\pointname{ Puntos}
	\qformat{Pregunta \thequestion \dotfill \thepoints}

	% ============================================
	% SECCIÓN 1: PREGUNTAS DE SELECCIÓN MÚLTIPLE (18 puntos)
	% ============================================
	
	\question[3]
	Observe el siguiente código:
	\begin{lstlisting}
	fp = open("texto.txt", "r")
	data1 = fp.read(10)
	data2 = fp.read(10)
	fp.close()
	\end{lstlisting}
	¿Qué afirmación es correcta sobre las variables \texttt{data1} y \texttt{data2}?\begin{choices}
	\choice Ambas contienen los primeros 10 caracteres del archivo
	\CorrectChoice \texttt{data1} contiene los primeros 10 caracteres y \texttt{data2} los siguientes 10 % CORRECTA
	\choice Ambas contienen todo el contenido del archivo
	\choice \texttt{data2} estará vacía porque el archivo ya fue leído
	\end{choices}
	
	% Justificación: El puntero del archivo avanza con cada operación de lectura
	
\begin{lstlisting}
	archivo = open("datos.txt", "w")
	archivo.write("Linea 1")
	archivo.write("Linea 2")
	archivo.close()
\end{lstlisting}

	
	¿Qué contendrá el archivo \texttt{datos.txt} después de ejecutar este código?
	
	\begin{choices}
		\choice Línea 1\\Línea 2 (en dos líneas separadas)
		\CorrectChoice Línea 1Línea 2 (en una sola línea) % CORRECTA
		\choice Solo ``Línea 2''
		\choice Genera un error
	\end{choices}
	
	% Justificación: write() no añade saltos de línea automáticamente
	
		\question[3]
	¿Cuál método se debe utilizar para leer una sola línea de un archivo de texto en Python?
	
	\begin{choices}
		\choice \texttt{read()}
		\CorrectChoice \texttt{readline()} % CORRECTA
		\choice \texttt{readlines()}
		\choice \texttt{readone()}
	\end{choices}
	
	% Justificación: readline() lee una línea completa hasta el carácter \n
	
	\question[3]
	¿Cuál es la principal ventaja de usar la sentencia \texttt{with} al trabajar con archivos?
	\begin{choices}
	\choice Permite leer archivos más rápido
	\CorrectChoice Cierra automáticamente el archivo al salir del bloque % CORRECTA
	\choice Permite abrir múltiples archivos simultáneamente
	\choice Convierte automáticamente los datos a listas
	\end{choices}
	
	% Justificación: with asegura el cierre automático del archivo, evitando fugas de recursos
	

	\question[3]
	¿Qué sucede si se intenta abrir un archivo en modo \texttt{'w'} cuando el archivo ya existe?
	\begin{choices}
	\choice Se genera un error
	\CorrectChoice El contenido anterior se elimina y se sobrescribe % CORRECTA
	\choice Los nuevos datos se agregan al final del archivo
	\choice El archivo se abre en modo solo lectura
	\end{choices}
	
	% Justificación: El modo 'w' siempre sobrescribe el archivo existente
	
	% ============================================
	% SECCIÓN 2: EJERCICIO PRÁCTICO 1 (10 puntos)
	% ============================================
	
	\question[19]
	Escriba un programa en Python que realice las siguientes tareas:
	\begin{enumerate}
	\item Lea un archivo de texto llamado \texttt{entrada.txt}
	\item Cuente el número total de palabras del archivo
	\item Cuente el número de palabras que tienen tilde sin importar mayúsculas o minúsculas
	\item Guarde los resultados en un archivo llamado \texttt{resultados.txt} con el siguiente formato:
	\end{enumerate}
	\begin{tcolorbox}[colback=gray!20, boxrule=0pt]
	Total de palabras: [número]\\
	Palabras con tilde: [número]\
	\end{tcolorbox}
	\textbf{Requisitos:}
	\begin{itemize}
	\item Use la sentencia \texttt{with} para manejar archivos
	\item Implemente manejo de errores para el caso en que el archivo no exista
	\item El código debe estar bien comentado
	\end{itemize}
	\textbf{Rúbrica:}
	\begin{itemize}
	\item Lectura correcta del archivo (3 puntos)
	\item Conteo de palabras correcto (4 puntos)
	\item Conteo de palabras con vocales correcto (6 puntos)
	\item Escritura correcta del archivo de resultados (5 puntos)
	\item Manejo de errores (1 punto)
	\end{itemize}
	% ============================================
	% SECCIÓN 3: EJERCICIO PRÁCTICO 2 (12 puntos)
	% ============================================
	
	\question[19]
	Se tiene un archivo CSV llamado \texttt{estudiantes.csv} con la siguiente estructura:
	\begin{tcolorbox}[colback=gray!20, boxrule=0pt]
	Nombre,Nota1,Nota2,Nota3\\
	Juan,4.5,3.8,4.2\\
	María,5.0,4.8,4.9\\
	Pedro,2.5,3.0,1.8
	\end{tcolorbox}
	Escriba un programa que:
	\begin{enumerate}
	\item Lea el archivo CSV
	\item Calcule el promedio de las tres notas para cada estudiante. Tenga en cuenta que el archivo podría tener hasta 100 estudiantes.
	\item Determine si el estudiante aprobó (promedio = 3.0)
	\item Cree un nuevo archivo CSV llamado \texttt{promedios.csv} con el formato:
	\end{enumerate}
	\begin{tcolorbox}[colback=gray!20, boxrule=0pt]
	Nombre,Promedio,Estado\\
	Juan,4.17,Aprobado\\
	María,4.90,Aprobado\\
	Pedro,2.43,Reprobado
	\end{tcolorbox}
	\textbf{Requisitos:}
	\begin{itemize}
	\item Use el módulo \texttt{csv} de Python
	\item Use \texttt{newline=''} al abrir los archivos CSV
	\item Redondee los promedios a 2 decimales
	\item Incluya manejo de errores
	\end{itemize}
	\textbf{Rúbrica:}
	\begin{itemize}
	\item Lectura correcta del CSV con encabezados (3 puntos)
	\item Cálculo correcto de promedios (6 puntos)
	\item Determinación correcta del estado (4 puntos)
	\item Escritura correcta del nuevo CSV (5 puntos)
	\item Manejo de errores y buenas prácticas (1 punto)
	\end{itemize}

	
	\end{questions}
	
	\end{multicols}
	\begin{center}
	    \gradetable[h][questions]
	\end{center}
	
	% NOTAS PARA EL PROFESOR
	% Total: 50 puntos
	% MCQ: 6 preguntas × 3 puntos = 18 puntos
	% Ejercicio 1: 10 puntos
	% Ejercicio 2: 12 puntos
	% Ejercicio 3: 10 puntos
	% Tiempo estimado: 100 minutos (15 min MCQ, 25 min Ej1, 35 min Ej2, 25 min Ej3)
	
\end{document}